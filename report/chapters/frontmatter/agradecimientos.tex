Este trabajo cierra una etapa académica muy importante en mi vida. Durante este máster en visión artificial he aprendido muchas cosas a nivel social, académico, laboral y emocional que me han servido para abrir la mente en muchos aspectos donde el grado se quedaba solamente en la superficie. He visto en el profesorado pasión en la enseñanza de su campo e intentaban transmitirlo de igual manera a nosotros. La gran mayoría han conseguido transmitirme ese sentimiento en el aprendizaje, estudio y posterior uso en el trabajo y espero en el futuro poder tener esa pasión y, más importante, saber transmitírsela a otras personas. Mi primer agradecimiento va para todos ellos por su esfuerzo, predisposición y dedicación.

Todo lo comentado en el párrafo anterior no hubiera pasado si no hubiera conocido al que fue mi profesor en el grado en varias asignaturas, tutor en el trabajo de fin de grado, jefe de un proyecto que hemos llevado desde la universidad, tutor de este trabajo, persona de confianza y amigo, Jose María Cañas. Agradecerle la infinita paciencia en cada una de las etapas enumeradas así como la paciencia en cada reunión donde su positivismo siempre vencía cuando no conseguíamos los objetivos planteados en cada reunión. Es un claro ejemplo de esa transmisión del entusiasmo y la pasión en su oficio que hace mella en muchos de nosotros cuando hemos compartido con él un pequeño o gran trabajo. Espero que sean muchos más. Agradecer también a Eduardo Perdices su predisposición para llevar junto a Jose María este trabajo e intentar darle la forma necesaria para encajarlo en el hueco que pretendíamos. La humildad y profesionalidad consiguen siempre ser un espejo al que mirar y al que querer pertenecer.

Como no, agradecer a los compañeros Carlos y Aitor que también se sumaron en esta aventura del máster de visión, de los cuales aprendí mucho estos años y de los cuales me llevo muy buenos conocimientos, aprendizaje y recuerdos. En especial a Fran, por acompañarme este año tan especial en la tontería y hacerlo más llevadero. No perdamos nunca ese humor.

La joya de la corona en esta página es para la persona que me lleva acompañando en todas las etapas de mi vida y la que, supongo, se pensaba que no iba a aparecer en este trabajo, Patricia. Solo por la paciencia que has tenido por las charlas sobre cómo debería aprender el robot, lo momentos de "no aprende" y por los interminables: "no puedo, tengo que seguir con el TFM" mereces aparecer en cada rincón de este documento. Tu carácter, humor y amor son el motor que me hace insistir en cada tarea que me propongo. Espero estar a la altura cuando sea al revés.