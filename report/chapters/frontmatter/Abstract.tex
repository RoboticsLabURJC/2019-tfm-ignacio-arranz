% CREATED BY DAVID FRISK, 2018

Departamento de Sistemas Telemáticos y Computación,\\
Universidad Rey Juan Carlos \setlength{\parskip}{0.5cm}

\thispagestyle{plain}			% Supress header 
\setlength{\parskip}{0pt plus 1.0pt}
\section*{Resúmen}
Este Trabajo de Fin de Máster describe el desarrollo llevado a cabo para la construcción de un entorno de pruebas donde un modelo de un Fórmula-1, a través de algoritmos de Aprendizaje por Refuerzo es capaz de resolver el ejercicio del Sigue-Líneas utilizando visión computacional. Mediante diferentes ensayos con diferentes configuraciones de parámetros de acciones y percepciones se hará un estudio de la configuración que mejor resuelve el circuito en términos de tiempos por vuelta comparado con un robot que lo resuelve de manera explícita. Toda esta infraestructura se hará utilizando estándares del software libre en el mundo de la robótica como son ROS, para las comunicaciones, Gazebo para la representación tridimensional de la escena y Python como lenguaje de programación.


% KEYWORDS (MAXIMUM 10 WORDS)
\vfill
Palabras clave: Aprendizaje por refuerzo, Q-Learning, robótica, visión artificial, conducción autónoma, Python.

\newpage				% Create empty back of side
\thispagestyle{empty}
\mbox{}